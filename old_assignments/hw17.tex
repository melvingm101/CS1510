\documentclass[10pt]{article}
\usepackage{fullpage}
\usepackage{graphicx}
\usepackage{amssymb}
\usepackage{qtree}
\usepackage{rotating}
\newcommand{\tab}{\hspace*{2em}}
\newcommand{\tabb}{\hspace*{4em}}
\newcommand{\tabbb}{\hspace*{6em}}
\newcommand{\tabbbb}{\hspace*{8em}}
\newcommand{\tabbbbb}{\hspace*{10em}}
\newcommand{\norm}[1]{\left|\left|#1\right|\right|}
\setlength{\parindent}{0in} 
\begin{document}
	\begin{flushright}
	Lindsey Bieda and Joe Frambach\\
	Dynamic Programming Problems\\
	10.12.2011
	\end{flushright}
	23.	Consider the problem where the input is a collection of $n$ train trips within Germany. For the $i$th trip
		$T_i$ you are given the date $d_i$ of that trip, and the non-discounted fare $f_i$ for that trip. The Germani
		railway system sells a Bahncard for $B$ Marks that entitles you to a 50\% fare reduction on all train travel
		within Germany within $L$ days of purchase. The problem is to determine when to buy a Bahncard to
		minimize the total cost of your travel.\\
		\\
		Tree definition:\\
		The tree is defined as being rooted at a ``null" state where no decisions have been made. At each level, $\ell$,
		we may or may not purchase a Bahncard on $d_\ell$. If a card is purchased then we add $B + \frac{f_\ell}{2}$ to
		the node's value which is storing the total cost of all the trips. If the previously purchased Bahncard is still
		in effect add $\frac{f_\ell}{2}$. Otherwise we add $f_\ell$. \\
		\\ 
		Pruning rules:
		\begin{enumerate}
			\item At each level $\ell$, there are multiple opportunities, $0$ to $\ell$, for the "last card purchased".
			Trips and cards purchased after level $\ell$ are independent of the prior history, so we can prune all
			but the history with minimum cost. More succinctly: at every level $\ell$, prune all but the cheapest
			travel history with the same last-card-purchased.\\
		\end{enumerate}
		Algorithm:\\
		$A[*,*] = \infty$\\
		$A[0,0] = 0.$ \emph{Inititalize. $A[level,last-purchase]$ = $total-cost$.}\\
		\\
		For $\ell$ = 0 to $n-1$:\\
		\tab For $p$ = 0 to $\ell$:\\
		\tabb A[$\ell+1$, $\ell+1$] = min(A[$\ell+1$,$\ell+1$], A[$\ell$,p] + B +  $\frac{f_{\ell+1}}{2}$) \emph{Purchasing a new card and getting half off.}\\
		\tabb If $\ell - p < \ell$: \emph{Previous card is still active get half off.}\\
		\tabbb A[$\ell$+1, p] = min( A[$\ell$+1, p], A[$\ell$, p] + $\frac{f_{\ell+1}}{2}$ )\\
		\tabb Else: \emph{Pay full price.}\\
		\tabbb A[$\ell$+1, p] = min( A[$\ell$+1, p], A[$\ell$, p] + $f_{\ell+1}$ )\\
		\\
		The solution will be found at min(A[$n$,*])\\
	\newpage
	1.	A square matrix $M$ is lower triangular if each entry above the main diagonal is zero, that is, each entry
		$M_{i,j}$ , with $i < j$, is equal to zero. Show that if there is an $O(n^2)$ time algorithm for multiplying two $n$
		by $n$ lower triangular matrices then there is an $O(n^2)$ time algorithm for multiplying two arbitrary $n$
		by $n$ matrices.\\
		\\
		Given two $nxn$ matrices, $M_1$ and $M_2$.\\
		First, decompose $M_1$ into a sum of lower- and upper- diagonal matrices $L_1$ and $U_1$ such that $M_1 = L_1 + U_1$.\\
		Decompose $M_2$ in the same manner, so that $M_2 = L_2 + U_2$.\\
		\[
		M_1M_2 = (L_1 + U_1)(L_2 + U_2)\\
		= L_1L_2 + L_1U_2 + U_1L_2 + U_1U_2\\
		\]
		Each component can be calculated in O($n^2$) time, assuming there is an $O(n^2)$ time algorithm for multiplying
		two $n$ by $n$ lower triangular matrices.\\
		\begin{enumerate}
		\item Calculating $L_1L_2$: These are already both lower-triangular.
		\item Calculating $L_1U_2$:
			\[
			\left[
				\begin{array}{c c}
					0 & 0\\
					0 & L_1
				\end{array}
			\right]
			\left[
				\begin{array}{c c}
					0 & 0\\
					U_2 & 0
				\end{array}
			\right]
			=
			\left[
				\begin{array}{c c}
					0 & 0\\
					L_1U_2 & 0
				\end{array}
			\right]
			\]
		
		\item Calculating $U_1L_2$:
			\[
			\left[
				\begin{array}{c c}
					0 & 0\\
					U_1 & 0
				\end{array}
			\right]
			\left[
				\begin{array}{c c}
					L_2 & 0\\
					0 & 0
				\end{array}
			\right]
			=
			\left[
				\begin{array}{c c}
					0 & 0\\
					U_1L_2 & 0
				\end{array}
			\right]
			\]
		\item Calculating $U_1U_2$:\\
		$U_1U_2 = {(U_2^T U_1^T)}^T$ 
		 
		\end{enumerate}
\end{document}
