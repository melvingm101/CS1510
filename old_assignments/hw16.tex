\documentclass[10pt]{article}
\usepackage{fullpage}
\usepackage{graphicx}
\usepackage{amssymb}
\usepackage{qtree}
\usepackage{rotating}
\newcommand{\tab}{\hspace*{2em}}
\newcommand{\tabb}{\hspace*{4em}}
\newcommand{\tabbb}{\hspace*{6em}}
\newcommand{\tabbbb}{\hspace*{8em}}
\newcommand{\tabbbbb}{\hspace*{10em}}
\newcommand{\norm}[1]{\left|\left|#1\right|\right|}
\setlength{\parindent}{0in} 
\begin{document}
	\begin{flushright}
	Lindsey Bieda and Joe Frambach\\
	Dynamic Programming Problems\\
	10.05.2011
	\end{flushright}
	20. The input to this problem is two sequences $T = t_1, \ldots, t_n$ and $P = p_1, \ldots, p_k$ such that $k = n$, and
	a positive integer cost $c_i$ associated with each $t_i$.  The problem is to find a subsequence of $T$ that
	matches $P$ with maximum aggregate cost. That is, find the sequence $i_1 < \ldots < i_k$ such that for all $j$,
	$1 = j = k$, we have $t_{i_j} = p_j$ and $\sum_{j=1}^k c_{i_j}$ is maximized.\\
	So for example, if $n = 5$, $T = XY~XXY$ , $k = 2$, $P = XY$, $c_1 = c_2 = 2$, $c_3 = 7$, $c_4 = 1$ and $c_5 = 1$, then
	the optimal solution is to pick the second $X$ in $T$ and the second $Y$ in $T$ for a cost of $7 + 1 = 8$.
	\begin{enumerate}
		\item[(a)]	Give a recursive algorithm to solve this problem. Then explain how to turn this recursive algorithm
					into a dynamic program.\\
					\\
					A function, Weighted Sub Sequence (WSS), is defined:\\
					The recursive algorithm is called initally as wss($n$,$k$) with $T$, $P$, and $C$ being globally accessable.\\
					The algorithm works by examining substrings of both of the given sequences and then determining where 
					the values at a given position are equal and maximizes the values at these positions. 
					\\
					wss(i,j):\\
					\tab if i = 0 or j = 0: \emph{outside the bounds of either string}\\
					\tabb return 0 \emph{no value here}\\
					\tab if i < j: \emph{if the length of P is less than the length of T there is no solution}\\
					\tabb return $- \infty$\\
					\tab else if $T_i$ = $P_j$: \emph{The last characters are equal. Either use it or ignore and continue.}\\
					\tabb return max($v_i$ + wss(i-1,j-1), wss(i-1,j)) \emph{check if there is a better location elsewhere in the string}\\
					\tab else:\\
					\tabb return wss(i-1,j) \emph{check the rest of the string}\\
					\\
					Given the above recursive definition we can draw a call tree and then determine what pruning rules
					to apply. From there we map the tree to an array based on these pruning rules.\\
					The pruning rules are based on $i$ and $j$ passed into the $WSS$ call, so the complexity
					is polynomial in terms of $n$ and $k$.\\
		\item[(b)]	Give a dynamic programming algorithm based on enumerating subsequences of $T$ and using the
					pruning method.\\
					\\
					Tree Defintion:\\
					Every node in the tree is a different substring of $T$. Each of the levels of the tree are created
					by either choosing to add the next character of the $T$ or not. We represent these states as 
					[level,number of matching characters] in the array, with the total cost being stored at that location.
					The tree is rooted at [0,0] representing the empty string with a total cost of 0.\\
					The children of $A[x,y]$ will be $A[x+1,y]$ and $A[x+1,y+1]$ (if $T_{x+1} == P_{y+1}$).\\
					\\
					Pruning Rules (Ruling Prunes):
					\begin{enumerate}
						\item[(1)] For every node on the same level with the same number of matching characters, prune all
						but the one with the maximum cost.
						\item[(2)] At every level, $\ell$, if the next added character to the substring does not increase the 
						amount of matching characters, prune that node.  
					\end{enumerate}
					Algorithm:\\
					wss:\\
					A[0 to n, 0 to k] = 0. \emph{Initialize costs to zero.}\\
					A[0,0] = 0. \emph{T's empty substring matches zero characters of P with a cost of zero}\\
					for $\ell$ = 0 to $n-1$: \emph{For each character in $T$}\\
					\tab for $m$ = 0 to $k-1$: \emph{For each amount of matching, to $k-1$, don't go past the end of P}\\
					\tabb $A[\ell+1,m]$ = max($A[\ell+1,m]$, $A[\ell,m]$)\\
					\tabb if $T_{\ell+1} == P_{m+1}$ then\\
					\tabbb $A[\ell+1,m+1]$ = max($A[\ell+1,m+1]$, $c_{\ell+1} + A[\ell,m])$)\\
					The solution is given at $A[n,k]$.
		\item[(c)]	Give a dynamic programming algorithm based on enumerating subsequences of $P$ and using the
					pruning method.\\
					\\
					Tree Defintion:\\
					Every node in the tree is a different substring of $T$. The nodes are represented as [level, last matching character index] 
					in the array, with the total cost being stored at that location. At each level a different substring of $P$ is being considered.
					With the level, $\ell$, representing the first $\ell$ characters of $P$. The tree is rooted at [0,0] representing the empty string.\\
					\\
					Pruning rules:
					\begin{enumerate}
						\item[(1)] At every level remove the generated substrings of $T$ that do not match the current considered substring of $P$ as defined 
						above.
						\item[(2)] At every level for nodes with the same last matching character index keep only the one with the maximum cost.
					\end{enumerate}
					Algorithm:\\
					wss:\\
					A[0 to k, 0 to n] = 0. \emph{Initalize costs to zero.}\\
					A[0,0] = 0. \emph{The empty string does not match anyting and has a cost of zero.}\\
					for $\ell$ = 0 to $k-1$ : \emph{For every level}\\
					\tab for $i$ = 0 to $n$: \emph{For every character in $T$}\\
					\tabb if $P_{\ell+1}$ == $T_{i}$:\\
					\tabbb A[$\ell+1$, $i$] = max(A[$\ell+1$, $i$], $c_{i}$ + A[$\ell$, i])\\
					\\
					The solution is given at max(A[$k$, *]). 
	\end{enumerate}
\end{document}
