\documentclass[10pt]{article}
\usepackage{fullpage}
\usepackage{graphicx}
\usepackage{amssymb}
\usepackage{qtree}
\usepackage{rotating}
\newcommand{\tab}{\hspace*{2em}}
\newcommand{\tabb}{\hspace*{4em}}
\newcommand{\tabbb}{\hspace*{6em}}
\newcommand{\tabbbb}{\hspace*{8em}}
\newcommand{\tabbbbb}{\hspace*{10em}}
\newcommand{\norm}[1]{\left|\left|#1\right|\right|}
\setlength{\parindent}{0in} 
\begin{document}
	\begin{flushright}
	Lindsey Bieda and Joe Frambach\\
	Dynamic Programming Problems\\
	10.19.2011
	\end{flushright}
	5. Show that if one of the following three problems has a polynomial time algorithm then they all do.
	\begin{itemize}
		\item 	The Independent Set Problem: The input is a graph $G$. The problem is to find the largest
				independent set in $G$. In an independent set all vertices are mutually nonadjacent.
		\item 	The Clique Problem: The input is a graph $G$. The problem is to find the largest clique in $G$. In
				a clique all vertices are mutually adjacent.
		\item	The Vertex Cover Problem: The input is a graph $G$. The problem is to find the smallest vertex
				cover in $G$. A set $S$ is a vertex cover if each edge in $G$ is incident to a vertex in $S$.
	\end{itemize}
	Independent Set $\leq$ Clique\\
	\\
	Program Independent Set:\\
	\tab read G\\
	\tab Construct $G^\prime$ with the same vertices\\
	\tab For each vertex pair: $(v_1, v_2) \ni v_1 \neq v_2$ \emph{This will take O($n^2$) time to check all pairs}\\
	\tabb If there is no edge from $v_1$ to $v_2$ in $G$:\\
	\tabbb Create edge from $v_1$ to $v_2$ in $G^\prime$\\
	\tab output Clique($G^\prime$)\\
	\\
	Since the independent set wants only non-adjacent nodes we can generate edges between all non-adjacent nodes and then
	find the clique of that in order to determine which ones are in the independent set, since the state of being a clique
	is the inverse of being an independent set. \\
	\\
	Clique $\leq$ Vertex Cover:\\
	\\
	Program Clique:\\
	\tab read $G$\\
	\tab Construct $G^\prime$ with the same vertices\\
	\tab For each vertex pair: $(v_1, v_2) \ni v_1 \neq v_2$ \emph{This will take O($n^2$) time to check all pairs}\\
	\tabb If there is no edge from $v_1$ to $v_2$ in $G$:\\
	\tabbb Create edge from $v_1$ to $v_2$ in $G^\prime$\\
	\tab $S$ = Set of vertices in $G$\\
	\tab $S^{\prime}$ = VertexCover($G^{\prime}$)\\
	\tab output $S$ - $S^{\prime}$\\
	\\
	Since the independent set of the inverse of $G$ will try to create the smallest independent set, we know that there must
	not be an edge between the two edges in that set. We also know that there must then be edges between all of the other remaining
	nodes and, therefore, there is a clique. 
	\\
	\\
	Vertex Cover $\leq$ Independent Set\\
	\\
	Program Vertex Cover:\\
	\tab read $G$\\
	\tab $S$ = IndependentSet($G$)\\
	\tab output $G$ - $S$ \emph{This will take O(n) time to perform set subtraction}\\
	\\
	Since we have the largest Independent Set we know that the items in this set all must not be adjacent to 
	each other, therefore, the smallest possible vertex cover must contain all vertices not in this set. If one
	vertices in this set was in the smallest vertex cover then there would be two adjacent nodes in the vertex
	cover and it would no longer be the smallest one.\\
	\\
	8. Show that the subset sum problem is self-reducible. The decision problem is to take a collection of
		positive integers $x_1, \ldots, x_n$ and an integer $L$ and decide if there is a subset of the $x_i$’s that sum to
		$L$. The optimization problem asks you to return the actual subset if it exits. So you must show that
		if the decision problem has a polynomial time algorithm then the optimization problem also has a
		polynomial time algorithm.\\
		\\
		subset sum optimization $\leq$ subset sum decision.\\
		\\
		program subset sub optimization:\\
		\tab read $x_1, \ldots, x_n$, $L$\\
		\tab If $x_1, \ldots, x_n$, $L$ decision is 1 then:\\
		\tabb S = [$x_1, \ldots, x_n$]\\ 
		\tabb foreach $x$ in $S$:\\
		\tabbb if $S \backslash x$, $L$ decision is 1 then: \emph{if the subset sum is still possible without x remove it}\\
		\tabbbb remove $x$ from $S$\\
		\tab output $S$\\
		\\
		We examine if each of the numbers is required to sum to the desired $L$. If it is not required then we can remove it 
		from the set.
	\newpage
	12. Consider the following 2Clique problem:\\
		INPUT: A undirected graph $G$ and an integer $k$.\\
		OUTPUT: 1 if $G$ has two vertex disjoint cliques of size $k$, and 0 otherwise.\\
		Show that this problem is $NP$-hard. Use the fact that the clique problem in $NP$-complete. The input
		to the clique problem is an undirected graph $H$ and an integer $j$. The output should be 1 if $H$ contains
		a clique of size $j$ and 0 otherwise. Note that a clique is a mutually adjacent collection of vertices. Two
		cliques are disjoint if they do not share any vertices in common.\\
		\\
		Clique $\leq$ 2-Clique\\
		\\
		Program Clique:\\
		\tab read $H$, $j$\\
		\tab G = H + H, where both copies of H are disjoint\\
		\tab output 2-Clique($G$,$j$)\\
		\\
		Therefore, 2Clique is NPH.\\
		\\
		If there is a clique in $H$, there must be two cliques in a graph made up of two disjoint copies of $H$.
\end{document}
