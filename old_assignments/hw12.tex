\documentclass[10pt]{article}
\usepackage{fullpage}
\usepackage{graphicx}
\usepackage{amssymb}
\usepackage{qtree}
\newcommand{\tab}{\hspace*{2em}}
\newcommand{\tabb}{\hspace*{4em}}
\newcommand{\tabbb}{\hspace*{6em}}
\setlength{\parindent}{0in} 
\begin{document}
	\begin{flushright}
	Lindsey Bieda and Joe Frambach\\
	Dynamic Programming Problems\\
	9.28.2011
	\end{flushright}
	\noindent
	12. 	Consider the code for the Knapsack program given in the class notes.
	\begin{enumerate}
	\item[(a)]	Explain how one can actually find the highest valued subset of objects, subject to the weight
							constraint, from the Value table computed by this code.\\
							\\
							From the table you can find the highest valued subset of objects by examining the table starting
							at $Value[k,S]$, where $k$ = number of objects considered and $S$ = the maximum weight. If
							$Value[k-1,S] = Value[k,S]$, then do not add object $k$ to the solution set and let $k = k - 1$. 
							Otherwise, add object $k$ to the solution set and let $S=S-w_k$, where $w_k = $
							the weight of object $k$, then let $k=k-1$. Repeat until $k=0$.
	\item[(b)]	Explain how to solve the Knapsack problem using only $O(L)$ memory/space and $O(nL)$ time. You
							need only find the value and weight of the optimal solution, not the actual collection of objects.\\
							\\
							In the iterative solution when building the table, the writes for column $k$ depend only on the values
							stored in column $k-1$. Therefore, our array only needs to have two columns. At each iteration (when $k$ increases),
							the columns alternate between being the read column and the write column. This eliminates copy time, although
							this isn't necessary. The alternating of read-column and write-column is a method taken from the idea of
							double-buffering.
	\end{enumerate}
	
	\noindent
	14.	Give an algorithm for the following problem whose running time is polynomial in $n + L$.\\
			Input: positive integers $v_1, \ldots, v_n$, with $L = \sum_{i=1}^n v_i$.\\
			Output: A solution (if one exists) to $\sum_{i=1}^n (-1)^{x_i} v_i = 0$ where each $x_i$ is either 0 or 1.\\
			\\
			\\
			Pruning rules:\\
			1) $-v_{h+1} - \ldots - v_n < current~sum < v_{h+1} + \ldots + v_n$, that is if the current sum is too high
			or too low to be negated by all remaining values, prune that sum.\\
			2) If there are duplicate values at the same level of the tree prune all but one.\\
			\\
			
			Begin with $v_1$ at the root of tree, because of the symmetric nature.\\
			A[1,$v_1$] = 0\\
			
		 	For h = 2 to n:\\
			\tab For $\ell$ = -L to L:\\
			\tabb A[h,\ell-$v_h$] = 1\\
			\tabb A[h,\ell+$v_h$] = 0\\
			\tabb L = L - $v_h$\\
			
			If there is a solution A[n,0] will be defined.\\
			Traceback: Begin at $A[\ell=n, s=0]$.\\
			Repeat:\\
				\tab If $A[\ell,s]$ = 1, then add $x_{\ell} = 1$ to the solution set and go to $A[\ell-1,s+v_{\ell}]$\\
				\tab Else, add $x_{\ell} = 0$ to the solution set and go to $A[\ell-1,s-v_{\ell}]$\\
			At the end of the last iteration, the current cell will be $A[1,v_1]$.\\
	
	\newpage
	15.	Give an algorithm for the following problem whose running time is polynomial in $n + log L$:\\
			Input: positive integers $v_1, \ldots , v_n$ and $L$.\\
			Output:  A solution (if one exists) to $(\sum_{i=1}^n x_i v_i)$ mod $n = L$ mod $n$ where each $x_i$ is either 0 or 1.\\
			Here $x$ mod $y$ means the remainder when $x$ is divided by $y$.\\
			Pruning rule(s):\\
			1) If there are duplicate values at the same level of the tree prune all but one.\\
			\\
			
			Begin with $v_1$ at the root of tree, because of the symmetric nature.\\
			A[1,$v_1$] = 1\\
			
		 	For h = 2 to n:\\
			\tab For $\ell$ = 0 to n-1:\\
			\tabb A[h,\ell] = 0\\
			\tabb A[h,(\ell+$v_h$) mod $n$] = 1\\
			
			If there is a solution $A[n,L$ mod $n]$ will be defined.\\
			Traceback: Begin at $A[\ell=n, s=L$ mod $n]$.\\
			Repeat:\\
				\tab If $A[\ell,s]$ = 0, then add $x_{\ell} = 0$ to the solution set and go to $A[\ell-1,s]$\\
				\tab Else, add $x_{\ell} = 1$ to the solution set and go to $A[\ell-1,(s-v_{\ell})$ mod $ n]$\\
			At the end of the last iteration, the current cell will be $A[1,v_1]$.\\	
\end{document}