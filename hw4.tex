\documentclass[10pt]{article}
\usepackage{fullpage}
\usepackage{graphicx}
\newcommand{\tab}{\hspace*{2em}}
\begin{document}
	\begin{flushright}
	Lindsey Bieda and Joe Frambach\\
	Greedy Algorithm Problems\\
	9.11.2011
	\end{flushright}
	\noindent
	8. Consider the following problem.  The input consists of $n$ skiers with heights $p_{1}, \ldots, p_{n}$, and $n$ skis
	with heights $s_{1}, \ldots, s{n}$. The problem is to assign each skier a ski to to minimize the average difference
	between the height of a skier and his/her assigned ski. That is, if the $i$th skier is given the $a(i)$th ski,
	then you want to minimize:
	
	\[ \frac{1}{n}\sum\limits_{i=1}^n|~p_{i} - s_{a(i)}~|\]
	\begin{enumerate}
		\item[(a)] Consider the following greedy algorithm. Find the skier and ski whose height difference is minimized. 
		Assign this skier this ski. Repeat the process until every skier has a ski. Prove of disprove
		that this algorithm is correct.\\
		\\
		% answer here
		This algorithm does not correctly solve the problem. Consider the following inputs:\\
		skiers = (0.3, 0.3, 0.5, 0.7, 0.7)\\
		ski heights = (0.4, 0.4, 0.4, 0.59, 0.6)\\
		\\
		$GRE = 0.09 + 0.1 + 0.1 + 0.1 + 0.3 = 0.59/5 = 0.118$\\
		$OPT = 0.1 + 0.1 + 0.1 + 0.11 + 0.1 =  0.51/5 = 0.102$\\
		\\
		\item[(b)] Consider the following greedy algorithm. Give the shortest skier the shortest ski, give the second
		shortest skier the second shortest ski, give the third shortest skier the third shortest ski, etc.
		Prove of disprove that this algorithm is correct.\\
		\\
		% answer here
		Prove that algorithm 8b is correct.\\
		\\
		Proof: $\exists$ input $I$ $\ni$ algorithm 8b will produce incorrect output.\\
		\\
		$GRE(I) = \{(G_{p_{1}}, G_{s_{1}}), \ldots, (G_{p_{n}},G_{s_{n}})\}$\\
		$OPT(I) = \{(O_{p_{1}}, O_{s_{1}}), \ldots, (O_{p_{n}},O_{s_{n}})\}$, where $OPT(I)$ agrees with $GRE(I)$ for the most steps and is sorted by increasing $O_{p_{n}}$, $k$\\
		Let $k$ be the first point of disagreement between algorithm 8b's output and the optimal output. 
		\\
		We generate $OPT^{\prime}$ by swapping (if necessary) $O_{p_{k}}$ with $O_{p_{k+i}}$, where $O_{p_{k+i}}$ = $G_{p_{k}}$ and swapping $O_{s_{k}}$ 
		with $O_{s_{k+i}}$, where $O_{s_{k+i}}$ = $G_{s_{k}}$. Since, greedy selects skiers and skis from smallest to the largest we know
		that $G_{p_{k}} \leq O_{p_{k}}$ and $G_{s_{k}} \leq O_{s_{k}}$, and since $OPT$ is sorted in increasing order for it to disagree with $GRE$ it must
		not be optimal $\bot$. 
	\end{enumerate}
	12.  We consider the following scheduling problem:\\
	\\
	INPUT: A collection of jobs $J_{1}, \ldots , J_{n}$, where the $i$th job is a tuple $(r_{i}, x_{i})$ of non-negative integers
	specifying the release time and size of the job.\\
	OUTPUT: A preemptive feasible schedule for these jobs on one processor that minimizes the total completion time 
	$\sum_{i=1}^n C_{i}$.\\
	\\
	A \textit{schedule} specifies for each unit time interval, the unique job that that is run during that time interval.
	In a \textit{feasible} schedule, every job $J_{i}$ has to be run for exactly $x_{i}$ time units after time $r_{i}$ . The \textit{completion}
	time $C_{i}$ for job $J_{i}$ is the earliest time when $J_{i}$ has been run for $x_{i}$ time units. Examples of these basic
	definitions can be found below.\\
	\\
	We consider two greedy algorithms for solving this problem that schedule times in an online fashion,
	that is the algorithms are of the following form:\\
	\\
	$t = 0$\\
	while there are jobs left not completely scheduled\\
	\tab Among those jobs $J_{i}$ such that $r_{i} \leq t$, and that have previously been scheduled for less\\
	\tab \tab than $x_{i}$ time units, pick a job $J_{m}$ to schedule at time $t$ according to some rule;\\
	\tab increment $t$\\
	\\
	One can get different greedy algorithms depending on the rule for selecting $J_{m}$. For each of the following
	greedy algorithms, prove or disprove that the algorithm is correct.  Proofs of correctness must use an
	exchange argument.\\
	\\
	\textbf{SJF:} Pick $J_{m}$ to be the job with minimal size $x_{i}$. Ties may be broken arbitrarily.\\
	\\
	\textbf{SRPT:} Let $y_{i,t}$ be the total time that job $J_{i}$ has been run before time $t$. Pick $J_{m}$ to be a job that has
	minimal remaining processing time, that it, that has minimal $x_{i} - y_{i,t}$. Ties may be broken arbitrarily.\\
	\\
	As an example of SJF and SRPT consider the following instance:  $J_{1} = (0, 100)$, $J_{2} = (10, 10)$ and
	$J_{3} = (1, 4)$. Both SJF and SRPT schedule job $J_{1}$ between time 0 and time 1, and job $J _{3}$ between time
	1 and time 5, when job $J_{3}$ completes, and job $J_{1}$ again between time 5 and time 10. At time 10, SJF
	schedules job $J_{2}$ because its original size 10 is less than job $J_{1}$'s original size 100.  At time 10, SRPT
	schedules job $J_{2}$ because its remaining processing time 10 is less than job $J_{1}$'s remaining processing
	time 94. Both SJF and SRPT schedule job $J_{2}$ between time 10 and 20, when $J_{2}$ completes, and then
	job $J_{1}$ from time 20 until time 114, which job $J_{1}$ completes.  Thus for both SJF and SRPT on this
	instance $C_{1} = 114$, $C_{2} = 20$ and $C_{3} = 5$ and thus both SJF and SRPT have total completion time 139.\\
	\\
	%answer here
	\\
	Consider the following jobs: $J = [(0,2), (0,3), (4,2)]$.\\
	SJF:
	$J_{1}$ runs from time 0 to 2, completing in $C_{1}$ = 2.
	$J_{2}$ runs from time 2 to 4.
	$J_{3}$ runs from time 4 to 6, completing in $C_{3}$ = 6.
	$J_{2}$ runs from time 6 to 7, completing in $C_{2}$ = 7.
	Total completion time for SJF is 15.\\
	SRPT:
	$J_{1}$ runs from time 0 to 2, completing in $C_{1}$ = 2.
	$J_{2}$ runs from time 2 to 5, completing in $C_{2}$ = 5.
	$J_{3}$ runs from time 5 to 7, completing in $C_{3}$ = 7.
	Total completion time is 14.\\
	Therefore, SJF is not optimal since SJF $<$ SRPT.\\
	\\
	\\
	Thm: SRPT(I) is correct.\\
	Proof: Assume $\exists$ input $I$ $\ni$ SRPT($I$) is incorrect.\\
	\\
	SRPT($I$) returns an ordered list of processes to run at each time interval.\\
	SRPT($I$) = $[J_{s(0)}, J_{s(1)}, ..., J_{s(n)}]$, where s(i) represents which job is to run at time i.\\
	OPT($I$)  = $[J_{o(0)}, J_{o(1)}, ..., J_{o(n)}]$, where o(i) represents which job is to run at time i, and OPT matches SRPT for the most number of steps.\\
	Let $k$ be the first step which OPT($I$) disagrees with SRPT($I$); OPT($I$)[j] == SRPT($I$)[j] $\forall j < k$.\\
	\\
	Let job $J_{s(k)}$ have completion time $C_{SRPT}(k)$ and let job $J_{o(k)}$ have completion time $C_{OPT}(k)$.\\
	Let job $J_{s(k)}$ have remaining time $R_{SRPT}(k)$ and let job $J_{o(k)}$ have remaining time $R_{OPT}(k)$.\\
	Since SRPT chose $J_{s(k)}$ rather than $J_{o(k)}$, $R_{SRPT}(k) < R_{OPT}(k)$.\\
	\\
	Construct $OPT^{\prime}$ by modifying $OPT$, allocating the next $R_{SRPT}(k)$ time to Job $J_{s(k)}$,
	followed by $R_{OPT}(k)$ time to Job $J_{o(k)}$, effectively swapping the positions of the jobs in OPT.\\
	\\
	
	If $C_{SRPT}(k)$ == $C_{OPT}(k)$:\\
	Then OPT can be considered equal to SRPT for step $k$.\\
	
	If $C_{SRPT}(k) > C_{OPT}(k)$:\\
	Then Job $J_{s(k)}$ will be able to finish earlier than $J_{o(k)}$ originally did,
	and Job $J_{o(k)}$ will finish at the same time as $J_{s(k)}$ originally did,
	since both jobs' cumulative finish times were unchanged by swapping.\\
	Therefore, in this case, $OPT^\prime > OPT$.\\
	
	If $C_{SRPT}(k) < C_{OPT}(k)$:\\
	Then Job $J_{o(k)}$ will be able to finish earlier than $J_{s(k)}$ originally did,
	and Job $J_{s(k)}$ will finish at the same time as $J_{o(k)}$ originally did,
	since both jobs' cumulative finish times were unchanged by swapping.\\
	Therefore, in this case, $OPT^\prime > OPT$.\\
	
	In all cases, $OPT^\prime > OPT$. $\bot$.\\
	
\end{document}