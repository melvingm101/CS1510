\documentclass[10pt]{article}
\usepackage{fullpage}
\usepackage{graphicx}
\usepackage{amssymb}
\usepackage{qtree}
\usepackage{rotating}
\newcommand{\tab}{\hspace*{2em}}
\newcommand{\tabb}{\hspace*{4em}}
\newcommand{\tabbb}{\hspace*{6em}}
\newcommand{\tabbbb}{\hspace*{8em}}
\newcommand{\tabbbbb}{\hspace*{10em}}
\newcommand{\norm}[1]{\left|\left|#1\right|\right|}
\setlength{\parindent}{0in} 
\begin{document}
	\begin{flushright}
	Lindsey Bieda and Joe Frambach\\
	Reduction and Parallel Problems\\
	11.11.2011
	\end{flushright}

12. Explain how to modify the all-pairs shortest path algorithm for a CREW PRAM that was given in
class so that it runs in time O(log
2
n) on a EREW PRAM with n
3
processors.

\newpage
13. Explain how to modify the all-pairs shortest path algorithm for a CREW PRAM that was given in
class so that it actually returns the shortest paths (not just their lengths) in time O(log
2
n) on a EREW
PRAM with n
3
processors.

\newpage
14. Explain how to solve the longest common subsequence problem in time O(log
2
n) using at most a
polynomial number of processors on a CREW PRAM.
HINT: One way to do this is to reduce the longest common subsequence problem to a shortest path
problem. Note that the shortest path algorithm works for any graph for which there are not cycles
whose aggregate weight is negative.

\newpage
16. Design a parallel algorithms that merges two sorted arrays into one sorted array in time O(1) using a
polynomial number of processors on a CRCW PRAM.


\end{document}
