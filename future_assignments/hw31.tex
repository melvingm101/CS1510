\documentclass[10pt]{article}
\usepackage{fullpage}
\usepackage{graphicx}
\usepackage{amssymb}
\usepackage{qtree}
\usepackage{rotating}
\newcommand{\tab}{\hspace*{2em}}
\newcommand{\tabb}{\hspace*{4em}}
\newcommand{\tabbb}{\hspace*{6em}}
\newcommand{\tabbbb}{\hspace*{8em}}
\newcommand{\tabbbbb}{\hspace*{10em}}
\newcommand{\norm}[1]{\left|\left|#1\right|\right|}
\setlength{\parindent}{0in} 
\begin{document}
	\begin{flushright}
	Lindsey Bieda and Joe Frambach\\
	Reduction and Parallel Problems\\
	11.16.2011
	\end{flushright}

19. Design a parallel algorithm that nds the maximum number in a sequence x1; : : : ; xn of (not necessarily
distinct) integers in the range 1 to n. Your algorithm should run in constant time on a CRCW
Commmon PRAM with n processors. Note that it is important here that the xi
's have restricted
range.


21. Give a parallel algorithm for the following problem that runs in time O(log n) on an EREW PRAM.
The input is a binary tree with n nodes. Assume that each processor has a pointer to a unique node in
the tree. The problem is to number the leafs consecutively from left to right (that is in-order). Note
that this algorithm is needed for the algorithm in the notes for computing arithmetic expressions.


23. Design a parallel algorithm that takes a binary expression tree, where the leaves are integers, and the
internal nodes are the four standard arithmetic operators addition, subtraction, multiplication, and
division, and computes the value of the expression. Your algorithm should run in O(log n) time on a
CREW PRAM with n processors, where n is the number of nodes in the tree. You may assume that
each processor initially has a pointer to a unique node in the tree.
HINT: Following the technique used for subtraction in the class notes you need only nd a collection
of functions that contain the identity function and constant functions, and is closed under addition,
subtraction, multiplication, division with constants, and composition.









\end{document}
