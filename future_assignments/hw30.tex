\documentclass[10pt]{article}
\usepackage{fullpage}
\usepackage{graphicx}
\usepackage{amssymb}
\usepackage{qtree}
\usepackage{rotating}
\newcommand{\tab}{\hspace*{2em}}
\newcommand{\tabb}{\hspace*{4em}}
\newcommand{\tabbb}{\hspace*{6em}}
\newcommand{\tabbbb}{\hspace*{8em}}
\newcommand{\tabbbbb}{\hspace*{10em}}
\newcommand{\norm}[1]{\left|\left|#1\right|\right|}
\setlength{\parindent}{0in} 
\begin{document}
	\begin{flushright}
	Lindsey Bieda and Joe Frambach\\
	Reduction and Parallel Problems\\
	11.14.2011
	\end{flushright}

15. Give an algorithm for the minimum edit distance problem that runs in poly-log time on a CREW
PRAM with with a polynomial number of processors. Here poly-log means O(log
k
n) where n is the
input size, and k is some constant independent of the input size.
Recall that the input to this problem is a pair of strings A = a1 : : : am and B = b1 : : : bn. The goal is
to convert A into B as cheaply as possible. The rules are as follows. For a cost of 3 you can delete any
letter. For a cost of 4 you can insert a letter in any position. For a cost of 5 you can replace any letter
by any other letter.


17. Design a parallel algorithm that nds the maximum number in a sequence x1; : : : ; xn of (not necessarily
distinct) integers. Your algorithm should run in time O(log log n) on a CRCW PRAM with n processors.
HINT: Recall that you can nd the maximum of k numbers in O(1) time with 
(k
2
) processors. Try
divide and conquer into
p
n subproblems.


18. Design a parallel algorithm that nds the maximum number in a sequence x1; : : : ; xn of (not necessarily
distinct) integers in the range 1 to n. Your algorithm should run in constant time on a CRCW Priority
PRAM with n processors. Note that it is important here that the xi
's have restricted range. In a
CRCW priority PRAM, each processor has a unique positive integer identier, and in the case of write
con
icts, the value written is the value that the processor with the lowest identier is trying to write.


20. Show that if there is an an algorithm for a particular problem that runs in time T (n; p) on a p
processor CRCW machine, then there is an algorithm for this problem that runs in time T (n; p) log p
on a p processor EREW machine,
Hint: So you need to come up with an algorithm for the following problems:
(a) At some particular time, some subcollection P of processors want to read from a particular location
x. The processors in P do not know the identity, or even the number of processors in P . Yet the
processor in P need to work together to learn the value stored in location x in time O(log p).
(b) At some particular time, some subcollection P of processors want to write a common value to
a particular location x. The processors in P do not know the identity, or even the number of
processors in P . Yet the processor in P need to work together to write the value to in location x
in time O(log p).


\end{document}
