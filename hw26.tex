\documentclass[10pt]{article}
\usepackage{fullpage}
\usepackage{graphicx}
\usepackage{amssymb}
\usepackage{qtree}
\usepackage{rotating}
\newcommand{\tab}{\hspace*{2em}}
\newcommand{\tabb}{\hspace*{4em}}
\newcommand{\tabbb}{\hspace*{6em}}
\newcommand{\tabbbb}{\hspace*{8em}}
\newcommand{\tabbbbb}{\hspace*{10em}}
\newcommand{\norm}[1]{\left|\left|#1\right|\right|}
\setlength{\parindent}{0in} 
\begin{document}
	\begin{flushright}
	Lindsey Bieda and Joe Frambach\\
	Reduction and Parallel Problems\\
	11.04.2011
	\end{flushright}

21. The input to the Fixed Hamiltonian path problem is an undirected graph $G$ and two vertices $x$ and $y$
in $G$. The problem is to determine if there is a simple path between $x$ and $y$ in $G$ that spans all the
vertices in $G$. A path is simple if it doesn't include any vertex more than once. Show that if the Fixed
Hamiltonian path problem has a polynomial time algorithm then the Hamiltonian cycle problem has
a polynomial time algorithm.\\
HINT: I think it is easier to do this problem if you don't restrict yourself to a many-to-one reduction,
that is, feel free to call the path procedure multiple times.

\newpage
25. We consider a generalization of the Fox, goose and bag of beans puzzle
\verb=http://en.wikipedia.org/wiki/Fox,_goose_and_bag_of_beans_puzzle= \\
The input is a graph $G$ an integer $k$. The vertices of $G$ are objects that the farmer has to transport
over the river, there are an edge between two objects if they can not be left alone together on the same
size of the river. The goal is to determine if a boat of size $k$ is sucient to safely transport the objects
across the river. The size of the boat is the number of objects that the farmer can haul in the boat.
Show that this problem is NP-hard using a reduction from one of the problems that either I showed
was NP-hard in class, or that you showed was NP-hard in the homework. So I am letting you pick the
problem to reduce from here. You should take some time to reflect which problem would be easiest to
reduce from.

\newpage
4. Design a parallel algorithm for the parallel prefix problem that runs in time O($log~n$) with $n/log~n$
processors on a EREW PRAM.

\newpage
5. Give an algorithm that given an integer $n$ computes $n!$, that is $n$ factorial, in time O($log~n$) on an
EREW PRAM with n processors. Make the unrealistic assumption that a word of memory can store
arbitrarily large integers.

\end{document}
