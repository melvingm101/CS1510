\documentclass[10pt]{article}
\usepackage{fullpage}
\usepackage{graphicx}
\usepackage{amssymb}
\usepackage{qtree}
\usepackage{rotating}
\newcommand{\tab}{\hspace*{2em}}
\newcommand{\tabb}{\hspace*{4em}}
\newcommand{\tabbb}{\hspace*{6em}}
\newcommand{\tabbbb}{\hspace*{8em}}
\newcommand{\tabbbbb}{\hspace*{10em}}
\newcommand{\norm}[1]{\left|\left|#1\right|\right|}
\setlength{\parindent}{0in} 
\begin{document}
	\begin{flushright}
	Lindsey Bieda and Joe Frambach\\
	Reduction and Parallel Problems\\
	11.11.2011
	\end{flushright}

12. Explain how to modify the all-pairs shortest path algorithm for a CREW PRAM that was given in
class so that it runs in time O($log^2n)$ on a EREW PRAM with $n^3$ processors.\\
\\
We need to first create $n$ copies of the input using the algorithm described in 3a. This requires
$n^3$ to do this copying process in log $n$ time. The processor requirement is $n^3$ since we need to 
iterate over all pairs of nodes ($n^2$ processors) and then perform the actual copy of the edge from
the first to the second node using $n$ processors.
\begin{verbatim}
Repeat log n times
    ParFor i = 1 to n do
        ParFor j = 1 to n do
            ParFor m = 1 to n do
                each processor has it's own copy of D to work with (created above)
                T[i,m,j] = min{D[i,j], D[i,m]+D[m,j]}
                D[i,j]   = min{T[1,1,j] ... T[1,n,j]}
            
\end{verbatim}  

\newpage
13. Explain how to modify the all-pairs shortest path algorithm for a CREW PRAM that was given in
class so that it actually returns the shortest paths (not just their lengths) in time $O(log^2 n)$ on a EREW
PRAM with $n^3$ processors.\\
\\
We need to first create $n$ copies of the input using the algorithm described in 3a. This requires
$n^3$ to do this copying process in log $n$ time. The processor requirement is $n^3$ since we need to 
iterate over all pairs of nodes ($n^2$ processors) and then perform the actual copy of the edge from
the first to the second node using $n$ processors.
\begin{verbatim}
Repeat log n times
    ParFor i = 1 to n do
        ParFor j = 1 to n do
            ParFor m = 1 to n do
                each processor has it's own copy of D to work with (created above)
                T[i,m,j]    = min{D[i,j], D[i,m]+D[m,j]}
                D[i,j]      = min{T[1,1,j] ... T[1,n,j]}
                M[i,j]      = index of min{T[1,1,j] ... T[1,n,j]}
                This new table stores the best intermediate vertex between i and j
\end{verbatim}  

We then create $n$ copies of both $D$ and $M$ using $n^3$ using the algorithm in 3a. 
\begin{verbatim}
    ParFor i = 1 to n do
        ParFor j = 1 to n do
            We have n processors remaining to find the shortest path between i and j.
            From this point we perform a binary search where one processor in O(1) time
            find the shortest path from i to j which may route through m = M[i,j].

            Then in O(1) time two processors will concurrently find the shortest paths
            from i to m and m to j and so on and so on. This search is O(log n) time.
            
            When the processor returns the results are concatenated by the parent processor.
\end{verbatim}


\newpage
14. Explain how to solve the longest common subsequence problem in time O($log^2n$) using at most a
polynomial number of processors on a CREW PRAM.\\
HINT: One way to do this is to reduce the longest common subsequence problem to a shortest path
problem. Note that the shortest path algorithm works for any graph for which there are not cycles
whose aggregate weight is negative.\\
\\



\newpage
16. Design a parallel algorithms that merges two sorted arrays into one sorted array in time O(1) using a
polynomial number of processors on a CRCW PRAM.\\
\\
The input is two lists,\\
A = [$a_1, a_2, a_3, a_4, \ldots, a_{n-3}, a_{n-2}, a_{n-1}, a_n$]\\
B = [$b_1, b_2, b_3, b_4, \ldots, b_{n-3}, b_{n-2}, b_{n-1}, b_n$]\\
\\
At a high level, the algorithm works by considering the range $(a_i, a_{i+1})$ and inserting the values
of B which fall within that range (using parallel MIN and MAX operations in O(1) time). This happens
concurrently for all adjacent pairs in $A$.\\
If $b_1 < a_1$, then consider ranges $(b_i, b_{i+1})$ and insert values of A, rather than the other way.\\
For simplicity, though, assume $a_1 < b_1$.\\
\\
In order to do the insertions in constant time, the input is assumed to be doubly linked lists, so
the insertion is simply a constant-time rearrangement of pointers.\\
\\
ParFor i = 1 to n:\\
\tab Find the smallest $b_j$ greater than $a_i$.\\
\tab Find the largest $b_k$ smaller than $a_{i+1}$.\\
\tab Insert $b_j \ldots b_k$ between $a_i$ and $a_{i+1}$.\\
\\
Algorithm for finding $b_j$ = smallest $B$ greater than $a_i$:\\
ParFor k = 1 to n:\\
\tab if $b_k > a_i$:\\
\tabb write T[i,k] = $b_k$\\
\tab else:\\
\tabb write T[i,k] = $\infty$\\
\tab Using the O(1) CRCW MIN program in the class notes, find MIN(T[i,*]).\\
\tab Return MIN(T[i,*]) and its index.\\
Algorithm for smallest $b_k$ smaller than $a_{i+1}$ is complementary.\\


\end{document}
