\documentclass[10pt]{article}
\usepackage{fullpage}
\usepackage{graphicx}
\usepackage{amssymb}
\usepackage{qtree}
\usepackage{rotating}
\newcommand{\tab}{\hspace*{2em}}
\newcommand{\tabb}{\hspace*{4em}}
\newcommand{\tabbb}{\hspace*{6em}}
\newcommand{\tabbbb}{\hspace*{8em}}
\newcommand{\tabbbbb}{\hspace*{10em}}
\newcommand{\norm}[1]{\left|\left|#1\right|\right|}
\setlength{\parindent}{0in} 
\begin{document}
	\begin{flushright}
	Lindsey Bieda and Joe Frambach\\
	Dynamic Programming Problems\\
	10.26.2011
	\end{flushright}
	6.  Show that if one of the following three problems has a polynomial time algorithm then they all do.
	\begin{itemize}
		\item The problem is to determine whether a Boolean Circuit (with gates NOT, binary AND, and
					binary OR) has some input that causes all of the output lines to be 1.  Assume that the fan-out
					(the number of gates that the output of a single gate can be fed into) of the gates in a circuit may
					be arbitrary.
		\item The problem is to determine whether a Boolean Circuit (with gates NOT, binary AND, and
					binary OR), and fan-out at most 2, has some input that causes all of the output lines to be 1.
		\item The problem is to determine whether a planar Boolean Circuit (with gates NOT, binary AND,
					and binary OR) has some input that causes all of the output lines to be 1. A circuit is planar if
					it can be laid on on the 2D plane so that no pair of lines cross (if you like, you can assume that
					the layout is part of the input).
	\end{itemize}
	10.		Show that the clique problem is self-reducible.  The decision problem is to take a graph $G$ and an
				integer $k$ and decide if $G$ has a clique of size $k$ or not. The optimization problem takes a graph $G$, and
				returns a largest clique in $G$.  So you must show that if the decision problem has a polynomial time
				algorithm then the optimization problem also has a polynomial time algorithm. Recall that a clique is
				a collection of mutually adjacent vertices.
	\\
	\\
	15.	Consider the following variant of the MST problem. The input consists of an undirected graph $G$ and
			an integer $k$. The problem is to find a spanning tree $T$ of $G$ such that the degree of each node in $T$ is
			at most $k$, or report that no such tree exists. Show by reduction that if this problem has a polynomial
			time algorithm then the Hamiltonian path problem has a polynomial time algorithm. The Hamiltonian
			path problem asks you to determine whether a graph has a simple path that spans the vertices.
\end{document}