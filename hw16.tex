\documentclass[10pt]{article}
\usepackage{fullpage}
\usepackage{graphicx}
\usepackage{amssymb}
\usepackage{qtree}
\usepackage{rotating}
\newcommand{\tab}{\hspace*{2em}}
\newcommand{\tabb}{\hspace*{4em}}
\newcommand{\tabbb}{\hspace*{6em}}
\newcommand{\tabbbb}{\hspace*{8em}}
\newcommand{\tabbbbb}{\hspace*{10em}}
\newcommand{\norm}[1]{\left|\left|#1\right|\right|}
\setlength{\parindent}{0in} 
\begin{document}
	\begin{flushright}
	Lindsey Bieda and Joe Frambach\\
	Dynamic Programming Problems\\
	10.05.2011
	\end{flushright}
	20. The input to this problem is two sequences $T = t_1, \ldots, t_n$ and $P = p_1, \ldots, p_k$ such that $k = n$, and
	a positive integer cost $c_i$ associated with each $t_i$.  The problem is to find a subsequence of $T$ that
	matches $P$ with maximum aggregate cost. That is, find the sequence $i_1 < \ldots < i_k$ such that for all $j$,
	$1 = j = k$, we have $t_{i_j} = p_j$ and $\sum_{j=1}^k c_{i_j}$ is maximized.\\
	So for example, if $n = 5$, $T = XY~XXY$ , $k = 2$, $P = XY$, $c_1 = c_2 = 2$, $c_3 = 7$, $c_4 = 1$ and $c_5 = 1$, then
	the optimal solution is to pick the second $X$ in $T$ and the second $Y$ in $T$ for a cost of $7 + 1 = 8$.
	\begin{enumerate}
		\item[(a)]	Give a recursive algorithm to solve this problem. Then explain how to turn this recursive algorithm
					into a dynamic program.\\
					\\
					wss(i,j):\\
					\tab if i = 0 or j = 0:\\
					\tabb return 0\\
					\tab if $T_i$ = $P_j$:\\	
					\tabb return max($v_i$ + wss(i-1,j-1), wss(i-1,j))\\
					\tab else:\\
					\tabb return wss(i-1,j)\\
					
		\item[(b)]	Give a dynamic programming algorithm based on enumerating subsequences of $T$ and using the
					pruning method. 
		\item[(c)]	Give a dynamic programming algorithm based on enumerating subsequences of $P$ and using the
					pruning method.
	\end{enumerate}
\end{document}
