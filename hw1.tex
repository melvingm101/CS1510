\documentclass[10pt]{article}
\usepackage{fullpage}
\begin{document}
	\begin{flushright}
	Lindsey Bieda and Joe Frambach\\
	Greedy Algorithm Problems\\
	9.1.2011
	\end{flushright}
	\noindent
	1. Consider the following problem:\\ 
	INPUT: A set $ S = \{(x_{i},y_{i})| 1\leq i \leq n\} $ of intervals over the real line.\\
	OUTPUT: A maximum cardinality subset $S^{\prime}$ of $S$ such that no pair of intervals in $S^{\prime}$ overlap. \\
	Consider the following algorithm:\\
	\\
	Repeat until $S$ is empty
	\begin{enumerate}
		\item Select the iterval $I$ that overlaps the least number of other intervals.
		\item Add $I$ to final solution set $S^{\prime}$.
		\item Remove all intervals from $S$ that overlap with $I$.
	\end{enumerate}
	\\
	Given the following input:  1-2, 1-3, 1-3, 2-4, 3-5, 4-6, 5-7, 5-7, 6-7.
	\\
	\noindent
	2. Consider the following Interval Coloring Problem.\\
	\\
	INPUT: A set $ S = \{(x_{i},y_{i})| 1\leq i \leq n\} $ of intervals over the real line. 
	Think of interval $(x_{i},y_{i}$ as being a request for a room for a class that meets from 
	time $x_{i}$ to time $y_{i}$.\\
	OUTPUT: Find an assignment of classes to rooms that uses the fewest number of rooms.\\
	\\
	Note that every room request must be honored and that no two classes can use a room at the
	same time. 

	\begin{enumerate}
		\item[(a)] Consider the following iterative algorithm. Assign as many classes as possible to the
		first room (we can do this using the greedy algorithm discussed in class, and in the class notes),
		then assign as many classes as possible to the second room, then assign as many classes as possible
		to the third room, etc. Does this algorithm solve the Interval Coloring Problem? Justify your answer.\\
		\\
		Let $s$ be the maximum number of intervals that overlap at one particular point in time. For each iteration
		of the algorithm the maximum number of non-overlapping times will be selected. From this we can assume that
		the maximum number of iterations in order for this algorithm to complete will be equivalent to $s$. As we 
		assign a single room in each iteration the maximum number of rooms will be $s$. This algorithm does solve 
		the Interval Coloring Problem as it is able to find the optimal solution.
		 
		\item[(b)] Consider the following algorithm. Process the classes in increasing order of start time. Assume
		that you are processing class $C$. If there is a room $R$ such that $R$ has been assigned to an earlier
		class, and $C$ can be assigned to $R$ without overlapping previously assigned classes, then assign $C$ to
		$R$. Otherwise, put $C$ in a new room. Does this algorithm solve the Interval Coloring Problem? Justify your anser.
		% answer here
	\end{enumerate}
\end{document}
