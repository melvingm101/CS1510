\documentclass[10pt]{article}
\usepackage{fullpage}
\usepackage{graphicx}
\usepackage{amssymb}
\usepackage{qtree}
\newcommand{\tab}{\hspace*{2em}}
\newcommand{\tabb}{\hspace*{4em}}
\newcommand{\tabbb}{\hspace*{6em}}
\begin{document}
	\begin{flushright}
	Lindsey Bieda and Joe Frambach\\
	Dynamic Programming Problems\\
	9.25.2011
	\end{flushright}
	\noindent
	8.  The input to this problem is a sequence S of integers (not necessarily positive). The problem is to find
			the consecutive subsequence of S with maximum sum. ``Consecutive" means that you are not allowed
			to skip numbers. For example if the input was
			
			\[ 12, -14, 1, 23, -6, 22, -34, 13 \]
			
			\noindent
			the output would be 1, 23, -6, 22. Give a linear time algorithm for this problem.\\
			\\
			% answer here
	9.	The input to this problem is a tree $T$ with integer weights on the edges. The weights may be negative,
			zero, or positive.  Give a linear time algorithm to find the shortest simple path in $T$.  The length of a
			path is the sum of the weights of the edges in the path. A path is simple if no vertex is repeated. Note
			that the endpoints of the path are unconstrained.\\
			\\
			% answer here
\end{document}