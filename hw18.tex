\documentclass[10pt]{article}
\usepackage{fullpage}
\usepackage{graphicx}
\usepackage{amssymb}
\usepackage{qtree}
\usepackage{rotating}
\newcommand{\tab}{\hspace*{2em}}
\newcommand{\tabb}{\hspace*{4em}}
\newcommand{\tabbb}{\hspace*{6em}}
\newcommand{\tabbbb}{\hspace*{8em}}
\newcommand{\tabbbbb}{\hspace*{10em}}
\newcommand{\norm}[1]{\left|\left|#1\right|\right|}
\setlength{\parindent}{0in} 
\begin{document}
	\begin{flushright}
	Lindsey Bieda and Joe Frambach\\
	Dynamic Programming Problems\\
	10.14.2011
	\end{flushright}
	24.	Give a polynomial time algorithm for the following problem. The input consists of a sequence $R =
		R_0 , \ldots , R_n$ of non-negative integers, and an integer $k$. The number $R_i$ represents the number of
		users requesting some particular piece of information at time $i$ ( say from a www server. If the
		server broadcasts this information at some time $t$, the requests of all the users who requested the
		information strictly before time $t$ are satisfied. The server can broadcast this information at most
		$k$ times. The goal is to pick the $k$ times to broadcast in order to minimize the total time (over all
		requests) that requests/users have to wait in order to have their requests satisfied.\\
		\\
		\textbf{Tree definition}:\\
		Each level represents a time step. At each time step, we may decide to broadcast or not. So, each node has two
		children. Each node is indexed by its level, $\ell$, and the number of broadcasts used so far, $b$.
		The nodes each store two values: the total wait time since the last broadcast, $\sigma$, and the
		total wait time, $\Sigma$. The tree is rooted at $[0,0]=[0,0]$, the state before any requests have come in.\\
		In the case of a broadcast, $\sigma$ is reset to the current level's user-count; else $\sigma$
		is increased by the current level's user-count. $\Sigma$ is then increased by the new value of $\sigma$.\\
		\\
		\textbf{Pruning Rules}:
		\begin{enumerate}
			\item	For all levels if any node has used more than $k$ broadcasts prune that node. 
			\item 	If there are more broadcasts remaining than remaining levels, prune that node. Since, it would be
					impossible to use $k$ broadcasts. 
		\end{enumerate} 
		\textbf{Algorithm}:\\
		$A[*,*] = \infty$ \emph{Initialization}\\
		$A[0,0] = 0$\\
		\\
		For $\ell = 0$ to $n-1$:\\
		\tab For $b$ = max($0$, $k-1$) to min($k$, $n-\ell$): \emph{Take care of the pruning rules}\\
		\tabb A[$\ell+1$, $b$] = A[$\sigma + R_{\ell+1}$, $\Sigma + \sigma + R_{\ell+1}$]\\
		\tabb A[$\ell+1$, $b+1$] = A[$R_{\ell+1}$, $\Sigma + R_{\ell+1}$]\\
		\\
	25.	Assume that you are given a collection $B_1 , \ldots, B_n$ of boxes. You are told that the weight in kilograms
		of each box is an integer between 1 and some constant $L$, inclusive. However, you do not know the
		specific weight of any box, and you do not know the specific value of $L$. You are also given a pan
		balance. A pan balance functions in the following manner. You can give the pan balance any two
		disjoint sub-collections, say $S_1$ and $S_2$, of the boxes. Let $|S_1|$ and $|S_2|$ be the cumulative weight of the
		boxes in $S_1$ and $S_2$, respectively. The pan balance then determines whether $|S_1| < |S_2|$, $|S_1| = |S_2|$,
		or $|S_1| > |S_2|$. You have nothing else at your disposal other than these $n$ boxes and the pan balance.
		The problem is to determine if one can partition the boxes into two disjoint sub-collections of equal
		weight. Give an algorithm for this problem that makes at most $O(n^2 L)$ uses of the pan balance. For
		partial credit, find an algorithm where the number of uses is polynomial in $n$ and $L$.\\
		\\
		\newpage
	2.	Show that if there is an $O(n^k )$, $k \geq 2$, time algorithm for inverting a nonsingular $n$ by $n$ matrix $C$
		then there is an $O(n^k)$ time algorithm for multiply two arbitrary $n$ by $n$ matrices $A$ and $B$.\\
		\\
		Matrix Multiplication $\leq$ Matrix Inversion.\\
		\\
		Program Matrix Multiplication:\\
		\tab read $A,B$\\
		\tab \[
		C =
		\left[
		\begin{array}{ccc}
			\mathrm{I} & A & 0\\
			0 & \mathrm{I} & B\\
			0 & 0 & \mathrm{I}
		\end{array} 
		\right]
		\]
		\tab $C^{-1} = Inversion(C)$\\
		\tab We know that:\\ 
		\[
		C^{-1} =
		\left[
		\begin{array}{ccc}
			\mathrm{I} & -A & AB\\
			0 & \mathrm{I} & -B\\
			0 & 0 & \mathrm{I}
		\end{array} 
		\right]
		\]
		\tab AB = top right ``ninth" of $C^{-1}$\\
		\tab Output AB

\end{document}	
