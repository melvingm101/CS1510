\documentclass[10pt]{article}
\usepackage{fullpage}
\usepackage{graphicx}
\usepackage{amssymb}
\usepackage{qtree}
\usepackage{rotating}
\newcommand{\tab}{\hspace*{2em}}
\newcommand{\tabb}{\hspace*{4em}}
\newcommand{\tabbb}{\hspace*{6em}}
\newcommand{\tabbbb}{\hspace*{8em}}
\newcommand{\tabbbbb}{\hspace*{10em}}
\newcommand{\norm}[1]{\left|\left|#1\right|\right|}
\setlength{\parindent}{0in} 
\begin{document}
	\begin{flushright}
	Lindsey Bieda and Joe Frambach\\
	Reduction and Parallel Problems\\
	11.02.2011
	\end{flushright}
19. Show by reduction that if the decision version of the SAT-CNF problem has a polynomial time algorithm
then the decision version of the 3-coloring problem has a polynomial time algorithm.
\newpage

20. In the dominating set problem the input is an undirected graph $G$, the problem is to find the smallest
dominating set in $G$. A dominating set is a collection $S$ of vertices with the property that every vertex
$v$ in $G$ is either in $S$, or there is an edge between a vertex in $S$ and $v$. Show that the dominating set
problem is NP-hard using a reduction from the vertex cover problem.
\newpage

23. In the disjoint paths problem the input is a directed graph $G$ and pairs $(s_1, t_1), \ldots, (s_k, t_k)$ of vertices.
The problem is to determine if there exist a collection of vertex disjoint paths between the pairs of
vertices (from each $s_i$ to each $t_i$). Show that this problem is NP-hard by a reduction from the 3SAT
problem. Note that this problem is not easy.\\
HINT: Construct one pair ($s_i, t_i)$ for each variable $x_i$ in your formula $F$ . Intuitively there will be
two possible paths between $s_i$ and $t_i$ depending on whether $x_i$ is true or false. There will be a
component/subgraph $D_j$ of $G$ for each clause $C_j$ in $F$. There will be three possible paths between the
$(s_i, t_i)$'s pairs for each $D_j$ . You want that it is possible to route any two of these paths (but not all
three) through $D_j$.
\newpage

2. You know that lots of famous computer scientists have tried to find a fast efficient parallel algorithm
for the following Boolean Formula Value Problem:\\
INPUT: A Boolean formula $F$ and a truth assignment $A$ of the variables in $F$.\\
OUTPUT: $1$ if $A$ makes $F$ true, and $0$ otherwise.\\
Moreover, most computer scientists believe that there is no fast efficient parallel algorithm for the
Boolean Value Problem. You want to find a fast efficient parallel algorithm for some new problem $N$.
After much effort you can not find a fast efficient parallel algorithm for $N$, nor a proof that $N$ does
not have a fast efficient parallel algorithm. How could you give evidence that finding a fast efficient
parallel algorithm for $N$ is at least a hard of a problem as finding a fast efficient parallel algorithm for
Boolean Formula Value problem? Be as specific as possible, and explain how convincing the evidence
is.\\
Note that ``fast and efficient'' means poly-log time with a polynomial number of processors. The term
``poly-log'' means bounded by O($log^kn$) for some constant $k$.
\newpage

3. Consider the problem of taking as input an integer $n$ and an integer $x$, and creating an array $A$ of $n$
integers, where each entry of $A$ is equal to $x$.\\

Give an algorithm runs in time O($log~n$) on a EREW PRAM using $n$ processors. What is the
efficiency of this algorithm?\\

Give an algorithm that runs in time O($log~n$) on a EREW PRAM using $n/log~n$ processors. What
is the efficiency of this algorithm?\\

Give an algorithm that runs in time O(1) on a CRCW PRAM using $n$ processors. What is the
efficiency of this algorithm?

\end{document}
