\documentclass[10pt]{article}
\usepackage{fullpage}
\usepackage{graphicx}
\usepackage{amssymb}
\usepackage{qtree}
\usepackage{rotating}
\newcommand{\tab}{\hspace*{2em}}
\newcommand{\tabb}{\hspace*{4em}}
\newcommand{\tabbb}{\hspace*{6em}}
\newcommand{\tabbbb}{\hspace*{8em}}
\setlength{\parindent}{0in} 
\begin{document}
	\begin{flushright}
	Lindsey Bieda and Joe Frambach\\
	Dynamic Programming Problems\\
	10.03.2011
	\end{flushright}
	13.	Our goal is now to consider the Knapsack problem, and develop a method for computing the actual
			items to be taken in $O(L)$ space and $O(nL)$ time.
	\begin{enumerate}
		\item[(a)]	Consider the following problem. The input is the same as for the knapsack problem, a collection
								of $n$ items $I_1, \ldots, I_n$ with weights $w_1, \ldots, w_n$, and values $v_1, \ldots, v_n$ , and a weight limit $L$.  The
								output is in two parts.  First you want to compute the maximum value of a subset $S$ of the $n$
								items that has weight at most $L$, as well as the weight of this subset.  Let us call this value and
								weight $v_a$ and $w_a$.  Secondly for this subset $S$ you want to compute the weight and value of the
								items in $\{I_1, \ldots, I_{n/2}\}$ that are in $S$. Let use call this value and weight $v_b$ and $w_b$. So your output
								will be two weights and two values. Give an algorithm for this problem that uses space $O(L)$ and
								time $O(nL)$.\\
							
							\begin{quotation}
								12. 	Consider the code for the Knapsack program given in the class notes.
								\begin{enumerate}
								\item[(a)]	Explain how one can actually find the highest valued subset of objects, subject to the weight
														constraint, from the Value table computed by this code.\\
														\\
														From the table you can find the highest valued subset of objects by examining the table starting
														at $Value[k,S]$, where $k$ = number of objects considered and $S$ = the maximum weight. If
														$Value[k-1,S] = Value[k,S]$, then do not add object $k$ to the solution set and let $k = k - 1$. 
														Otherwise, add object $k$ to the solution set and let $S=S-w_k$, where $w_k = $
														the weight of object $k$, then let $k=k-1$. Repeat until $k=0$.
								\item[(b)]	Explain how to solve the Knapsack problem using only $O(L)$ memory/space and $O(nL)$ time. You
														need only find the value and weight of the optimal solution, not the actual collection of objects.\\
														\\
														In the iterative solution when building the table, the writes for column $k$ depend only on the values
														stored in column $k-1$. Therefore, our array only needs to have two columns. At each iteration (when $k$ increases),
														the columns alternate between being the read column and the write column. This eliminates copy time, although
														this isn't necessary. The alternating of read-column and write-column is a method taken from the idea of
														double-buffering.
								\end{enumerate}			
							\end{quotation}
							
							From 12b, we take the resulting array and then iterate as follows to determine the set of items:\\
							$\ell$ = 0, $v_a$ = 0, $w_a$ = 0, $v_b$ = 0, $w_b$ = 0\\
							\tab For i = 1 to L:\\
							\tabb if $\exists~k$ = Array[i] - $\ell$:\\
							\tabbb add item $k$ to the solution set $S$, and $\ell$ = Array[i]\\
							\tabbb $v_a$ = $v_a$ + $v_k$\\ 
							\tabbb $w_a$ = $w_a$ + $w_k$\\
							\tabbb if $k \leq n/2$:\\
							\tabbbb $v_b$ = $v_b$ + $v_k$\\ 
							\tabbbb $w_b$ = $w_b$ + $w_k$\\
							
							
		\item[(b)]	Explain how to use the algorithm from the previous subproblem to get a divide and conquer
					algorithm for finding the items in the Knapsack problem a and uses space $O(L)$ and time $O(nL)$. \\
					We can partition the space for the knapsack problem by considering each of the items alone, and then 
					merging those solutions together in pairs and then continue merging until we have one full set. The 
					process of combining the solution sets uses the process mentioned above as determining which items
					make up the solution set. 
	\end{enumerate}
	\newpage
	\noindent
	16.	Give an algorithm for the following problem whose running time is polynomial in $n + W$:\\ 
			Input: positive integers $w_1, \ldots, w_n$ , $v_1, \ldots, v_n$ and $W$.\\
			Output:  The maximum possible value of $\sum_{i=1}^n x_i v_i$ subject to $\sum_{i=1}^n x_i w_i \leq W$ and each $x_i$ is a
			nonnegative integer.\\
			\\
			Pruning Rules:\\
			1) At each level, there may be no more than $W$ configurations. Ignore nodes where the cumulative sum is greater than $W$.\\
			2) At each level, multiple children are created. Only create $W/w_k$ children, since more would have a weight greater than $W$.\\
			3) If $\exists$ two or mode nodes on the same level with the same weight, prune all but the one with the most value.\\
			\\
			Begin with root node $A[k=0, S=0, x=0]$, where k is the \textit{level} of the tree, for $0$ to $n$, and $S$ is the cumulative weight, for $0$ to $W$, and x
			is the multiplier.\\
			For $k~=~0~to~n-1$:\\
			\tab For $S~=~0~to~W$:\\
			\tabb For $x~=~0~to~W/w_k$:\\
			\tabbb If $A[k,~ A[k,S,x], x] + v_k \cdot x  > \forall V \in A[k,~ A[k,S,x]]:$\\ 
			\tabbbb $A[k+1,~A[k,S,x]~+~x\cdot w_k, x]= A[k,~ A[k,S,x], x] + v_k \cdot x$.\\
			\newpage
	\newpage
	\noindent
	17. Give an algorithm for the following problem whose running time is polynomial in $n + L$, where \\
			$L =$ max $(\sum_{i=1}^n v_i^3, \prod_{i=1}^n v_i)$.\\
			Input: positive integers $v_1, \ldots, v_n$\\
			Output: A subset $S$ of the integers such that $\sum_{v_i \in S} v_i^3 = \prod_{v_i \in S} v_i$.
	
\end{document} 