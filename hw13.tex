\documentclass[10pt]{article}
\usepackage{fullpage}
\usepackage{graphicx}
\usepackage{amssymb}
\usepackage{qtree}
\newcommand{\tab}{\hspace*{2em}}
\newcommand{\tabb}{\hspace*{4em}}
\newcommand{\tabbb}{\hspace*{6em}}
\setlength{\parindent}{0in} 
\begin{document}
	\begin{flushright}
	Lindsey Bieda and Joe Frambach\\
	Dynamic Programming Problems\\
	9.28.2011
	\end{flushright}
	13.	Our goal is now to consider the Knapsack problem, and develop a method for computing the actual
			items to be taken in $O(L)$ space and $O(nL)$ time.
	\begin{enumerate}
		\item[(a)]	Consider the following problem. The input is the same as for the knapsack problem, a collection
								of $n$ items $I_1, \ldots, I_n$ with weights $w_1, \ldots, w_n$, and values $v_1, \ldots, v_n$ , and a weight limit $L$.  The
								output is in two parts.  First you want to compute the maximum value of a subset $S$ of the $n$
								items that has weight at most $L$, as well as the weight of this subset.  Let us call this value and
								weight $v_a$ and $w_a$.  Secondly for this subset $S$ you want to compute the weight and value of the
								items in $\{I_1, \ldots, I_{n/2}\}$ that are in $S$. Let use call this value and weight $v_b$ and $w_b$. So your output
								will be two weights and two values. Give an algorithm for this problem that uses space $O(L)$ and
								time $O(nL)$.
		\item[(b)]	Explain how to use the algorithm from the previous subproblem to get a divide and conquer
								algorithm for finding the items in the Knapsack problem a and uses space $O(L)$ and time $O(nL)$. 
	\end{enumerate}
	
	16.	Give an algorithm for the following problem whose running time is polynomial in $n + W$:\\
			Input: positive integers $w_1, \ldots, w_n$ , $v_1, \ldots, v_n$ and $W$.\\
			Output:  The maximum possible value of $\sum_{i=1}^n x_i v_i$ subject to $\sum{i=1}^n x_i w_i \leq W$ and each $x_i$ is a
			nonnegative integer.\\
	
	17. Give an algorithm for the following problem whose running time is polynomial in $n + L$, where \\
			$L =$ max $(\sum_{i=1}^n v_i^3, \prod_{i=1}^n v_i)$.\\
			Input: positive integers $v_1, \ldots, v_n$\\
			Output: A subset $S$ of the integers such that $\sum_{v_i \in S} v_i^3 = \prod_{v_i \in S} v_i$.
	
\end{document} 