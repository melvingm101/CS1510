\documentclass[10pt]{article}
\usepackage{fullpage}
\usepackage{graphicx}
\usepackage{amssymb}
\newcommand{\tab}{\hspace*{2em}}
\begin{document}
	\begin{flushright}
	Lindsey Bieda and Joe Frambach\\
	Dynamic Programming Problems\\
	9.20.2011
	\end{flushright}
	\noindent
	5. The input to this problem is a pair of strings $A = a_1 \ldots a_m$ and $B = b_1 \ldots b_n$. The goal is to convert
	$A$ into $B$ as cheaply as possible. The rules are as follows. For a cost of 3 you can delete any letter.
	For a cost of 4 you can insert a letter in any position. For a cost of 5 you can replace any letter by
	any other letter. For example, you can convert $A = abcabc$ to $B = abacab$ via the following sequence:
	$abcabc$ at a cost of 5 can be converted to $abaabc$, which at cost of 3 can be converted to $ababc$, which
	at cost of 3 can be converted to $abac$, which at cost of 4 can be converted to $abacb$, which at cost of 4
	can be converted to $abacab$. Thus the total cost for this conversion would be 19. This is almost surely
	not the cheapest possible conversion.\\
	\\
	% answer here
	\newpage
	\noident
	6. Find the optimal binary search tree for keys $K_1 < K_2 < K_3 < K_4 < K_5$ where the access probabilities/weights 
	are .5, .05, .1, .2, .25 respectively using the algorithm discussed in class and in the notes.
	Construct one table showing the optimal expected access time for all subtrees considered in the algorithm, 
	and another showing the roots of the optimal subtrees computed in the other table. Show how
	to use the table of roots to recompute the tree.\\
	\\
	% answer here
	$Optimal~expected~access~time~for~all~subtrees = 
	\begin{tabular}{|c|c|c|c|c|c|}
		\hline
		  & 1 & 2  & 3 & 4 & 5\\ \hline
		1 & 0.05 &	0.2 & 0.55 & 1.05 & 2.15\\ \hline
		2 & 0 & 0.1 & 0.4 & 0.9 & 1.95 \\ \hline
		3 & 0 & 0 & 0.2 & 0.65 & 1.6\\ \hline
		4 & 0 & 0 & 0 & 0.25 & 1\\ \hline
		5 & 0 & 0 & 0 & 0 & 0.5\\ \hline	
	\end{tabular}
	$\\
	\\
	$Roots~of~the~optimal~subtrees = 
	\begin{tabular}{|c|c|c|c|c|c|}
		\hline
		  & 1 & 2 & 3 & 4 & 5\\ \hline
		1 & 1 &	1 & 3 & 3 & 5\\ \hline
		2 &   & 2 & 3 & 3 & 4 \\ \hline
		3 &   &   & 3 & 4 & 5\\ \hline
		4 &   &   &   & 4 & 5\\ \hline
		5 &   &   &   &   & 5\\ \hline	
	\end{tabular}
	$\\
	\\
	Beginning at point (1,5) in the table, where l=1 and r=5, in the roots table we 
	determine the root to be $k_5$, where $k=5$, the left child is recursively defined 
	by repeating this procedure at (l, k-1) and the right child by (k+1, r). Using this
	method we get the following tree:
	\end{document}