\documentclass[10pt]{article}
\usepackage{fullpage}
\usepackage{graphicx}
\usepackage{amssymb}
\usepackage{qtree}
\usepackage{rotating}
\newcommand{\tab}{\hspace*{2em}}
\newcommand{\tabb}{\hspace*{4em}}
\newcommand{\tabbb}{\hspace*{6em}}
\newcommand{\tabbbb}{\hspace*{8em}}
\newcommand{\tabbbbb}{\hspace*{10em}}
\newcommand{\norm}[1]{\left|\left|#1\right|\right|}
\setlength{\parindent}{0in} 
\begin{document}
	\begin{flushright}
	Lindsey Bieda and Joe Frambach\\
	Dynamic Programming Problems\\
	10.28.2011
	\end{flushright}
	11.  Show that the Vertex Cover Problem is self-reducible. The decision problem is to take a graph $G$ and
an integer $k$ and decide if $G$ has a vertex cover of size $k$ or not.  The optimization problem takes a
graph $G$, and returns a smallest vertex cover in $G$.  So you must show that if the decision problem
has a polynomial time algorithm then the optimization problem also has a polynomial time algorithm.
Recall that a vertex cover is a collection $S$ of vertices with the property that every edge is incident to
a vertex in $S$.\\
    \\
	\textbf{VERTEXCOVER-OPT} $\leq$ \textbf{VERTEXCOVER-DEC}\\
	\\
	Program VERTEXCOVER-OPT:\\
	\tab Read graph $G$, integer $k$.\\
	\tab \emph{First, find the smallest size of the smallest vertex cover.}\\
	\tab For $k$ = 1 to $n$:\\
	\tabb If VERTEXCOVER-DEC($G$,$k$) returns true:\\
	\tabbb size $s$ = $k$.\\
	\tabbb End for-loop.\\
	\tab \emph{Next, find the actual cover by removing the vertices NOT in the cover}\\
	\tab For each vertex $v$ in $G$:\\
	\tab  $G^\prime$ = $G$ - $v$\\
	\tabb If VERTEXCOVER-DEC($G^\prime$, $s$) returns true:\\
	\tabbb $G$ = $G^\prime$.\\
	\tab \emph{G is the graph containing only the smallest cover.}\\
	\tab Return $G$.\\
    \\
	\newpage
	14.  Consider the problem where the input is a collection of linear inequalities.  For example, the input
might look like $3x-2y \leq 3$ and $2x-3y \geq 9$. The problem is to determine if there is an integer solution
that simultaneously satisfies all the inequalities. Show that this problem is  NP-hard using the fact that
it is NP-hard to determine if a Boolean formula in conjunctive normal form is satisfiable.
	\\
	\\
	\textbf{CNF-SAT} $\leq$ \textbf{LINEAR-INEQ}\\
	\\
	Program CNF-SAT:\\
	\tab Read formula $F$\\
	\tab $F^\prime$ = ConstructEquation($F$)\\ 
	\tab Output	LINEAR-INEQ($F^\prime$)\\
	\\
	ConstructEquation($F$):\\
	Separate the CNF formula by the conjuncts to create $n$ separate statements.\\
	For each of these $n$ statements constuct one side of the inequality by 
	summing each of the positive literals (now variables) and for the negated literals
	make them (1 - $literal$). 
	Make this sum $\geq 1$. This is done to enforce that at least one of the literals
	must be true in order to satisfy this part of the formula.\\
	\\
	For each of the literals add the following inequality $0 \leq literal \leq 1$. The
	purpose of this is to ensure that each literal can only have a value of 0 or 1 relating
	to the possibility of it bring true or false in the formula.  
	\\
	Ex: $x \vee y \vee \neg z$ becomes:\\ 
	$x + y + (1 - z) \geq 1$ and\\
	$0 \leq x \leq 1$,\\
	$0 \leq y \leq 1$,\\
	$0 \leq z \leq 1$\\
	\\
	The fact that the formula is satifiable if and only if the linear inequality has a solution that satisfies all
	inequalities is obvious from the above construction.\\
	\\
	\newpage		
	16.  The input to the three coloring problem is a graph $G$, and the problem is to decide whether the vertices
of $G$ can be colored with three colors such that no pair of adjacent vertices are colored the same color.
The input to the four coloring problem is a graph $G$, and the problem is to decide whether the vertices
of $G$ can be colored with four colors such that no pair of adjacent vertices are colored the same color.
Show by reduction that if the four coloring problem has a polynomial time algorithm then so does the
three coloring problem.\\
	\\
	\textbf{3-Coloring} $\leq$ \textbf{4-Coloring}\\
	\\
\end{document}
