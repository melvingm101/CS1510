\documentclass[10pt]{article}
\usepackage{fullpage}
\usepackage{graphicx}
\usepackage{amssymb}
\usepackage{qtree}
\usepackage{rotating}
\newcommand{\tab}{\hspace*{2em}}
\newcommand{\tabb}{\hspace*{4em}}
\newcommand{\tabbb}{\hspace*{6em}}
\newcommand{\tabbbb}{\hspace*{8em}}
\newcommand{\tabbbbb}{\hspace*{10em}}
\newcommand{\norm}[1]{\left|\left|#1\right|\right|}
\setlength{\parindent}{0in} 
\begin{document}
	\begin{flushright}
	Lindsey Bieda and Joe Frambach\\
	Parallel Problems\\
	11.09.2011
	\end{flushright}

9. The input to this problem is a character string $C$ of $n$ letters. The problem is to find the largest $k$
such that
\[C[1]C[2] \ldots C[k] = C[n - k + 1] \ldots C[n - 1]C[n]\]
That is, $k$ is the length of the longest prefix that is also a suffix. Give a EREW parallel algorithm that
runs in poly-logarithmic time with a polynomial number of processors.

\newpage
10. The input to this problem is a character string $C$ of $n$ letters. The problem is to find the largest $k$
such that
\[C[1]C[2] \ldots C[k] = C[n - k + 1] \ldots C[n - 1]C[n]\]
That is, $k$ is the length of the longest prefix that is also a suffix. Give a CRCW parallel algorithm that
runs in constant time with a polynomial number of processors.

\newpage
11. Design a parallel algorithm for adding two $n$-bit integers. You algorithm should run in O($log~n$) time
on a CREW PRAM with n processors.\\
NOTE: If your algorithm is EREW, you might want to rethink since I don't know how to do this easily
with out CR.\\
HINT: Use divide and conquer and generalize the induction hypothesis.

\end{document}
