\documentclass[10pt]{article}
\usepackage{fullpage}
\usepackage{graphicx}
\usepackage{amssymb}
\usepackage{qtree}
\usepackage{rotating}
\newcommand{\tab}{\hspace*{2em}}
\newcommand{\tabb}{\hspace*{4em}}
\newcommand{\tabbb}{\hspace*{6em}}
\newcommand{\tabbbb}{\hspace*{8em}}
\newcommand{\tabbbbb}{\hspace*{10em}}
\newcommand{\norm}[1]{\left|\left|#1\right|\right|}
\setlength{\parindent}{0in} 
\begin{document}
	\begin{flushright}
	Lindsey Bieda and Joe Frambach\\
	Parallel Problems\\
	11.09.2011
	\end{flushright}

9. The input to this problem is a character string $C$ of $n$ letters. The problem is to find the largest $k$
such that
\[C[1]C[2] \ldots C[k] = C[n - k + 1] \ldots C[n - 1]C[n]\]
That is, $k$ is the length of the longest prefix that is also a suffix. Give a EREW parallel algorithm that
runs in poly-logarithmic time with a polynomial number of processors.\\
\\
This solution relies heavily on the solution to problem 3a: using $n$ processors to create $n$ copies of 
an input $k$ in O($log~n$) time. I will just use the result as if it were a simple operation.\\
\\
The input is: $c_1, c_2, c_3, \ldots, c_{n-2}, c_{n-1}, c_n$.\\
We can use $n^2$ processors to create $n$ copies of the input in O($log~n$) time:\\
\[
\begin{array}{ccccccc}
c_1 & c_2 & c_3 & \ldots & c_{n-2} & c_{n-1} & c_n \\
c_1 & c_2 & c_3 & \ldots & c_{n-2} & c_{n-1} & c_n \\
    &     &     & \vdots &         &         &     \\
c_1 & c_2 & c_3 & \ldots & c_{n-2} & c_{n-1} & c_n \\
c_1 & c_2 & c_3 & \ldots & c_{n-2} & c_{n-1} & c_n \\
\end{array}
\]
Then, we do that again, except each processor has an ``offset'' 1 to $n$ for the write location.
The figure below should be clear.\\
\[
\begin{array}{ccccccccccccc}
    & c_1 & c_2 & \ldots & c_{n-3} & c_{n-2} & c_{n-1} & c_n \\
c_1 & c_2 & c_3 & \ldots & c_{n-2} & c_{n-1} & c_n \\
\\
    &     & c_1 & \ldots & c_{n-4} & c_{n-3} & c_{n-2} & c_{n-1} & c_n \\
c_1 & c_2 & c_3 & \ldots & c_{n-2} & c_{n-1} & c_n \\
\\
    &     &     & \vdots &         &         &     \\
\\
    &     &     &        &         & c_1     & c_2 & c_3 & \ldots & c_{n-1} & c_n \\
c_1 & c_2 & c_3 & \ldots & c_{n-2} & c_{n-1} & c_n \\
\\
    &     &     &        &         &         & c_1 & c_2 & \ldots & c_{n-1} & c_n \\
c_1 & c_2 & c_3 & \ldots & c_{n-2} & c_{n-1} & c_n \\
\end{array}
\]
Next, we use $n$ processors on each ``pairing'' above, for a total of $n^2$ processors. Each processor writes
a $1$ if the aligning characters are equal, otherwise write $0$.\\
\\
Next, we have $n$ AND-problems. Each pairing contains $1$ to $n$ ones or zeroes. If a prefix equals a suffix,
then all comparisons yielded $1$. The AND-problem for $n$ inputs is solved with $n$ processors in $log~n$ time (Problem 1).
\\
Now, we have to find the MAX of the alignment-lengths of the pairings which yielded $1$ for the comparison. This is again
done in $log~n$ time with $n$ processors using the same method as every other problem we've seen so far.\\


\newpage
10. The input to this problem is a character string $C$ of $n$ letters. The problem is to find the largest $k$
such that
\[C[1]C[2] \ldots C[k] = C[n - k + 1] \ldots C[n - 1]C[n]\]
That is, $k$ is the length of the longest prefix that is also a suffix. Give a CRCW parallel algorithm that
runs in constant time with a polynomial number of processors.\\
\\
Consider the following setup from the previous problem:\\
\[
\begin{array}{ccccccccccccc}
    & c_1 & c_2 & \ldots & c_{n-3} & c_{n-2} & c_{n-1} & c_n \\
c_1 & c_2 & c_3 & \ldots & c_{n-2} & c_{n-1} & c_n \\
\\
    &     & c_1 & \ldots & c_{n-4} & c_{n-3} & c_{n-2} & c_{n-1} & c_n \\
c_1 & c_2 & c_3 & \ldots & c_{n-2} & c_{n-1} & c_n \\
\\
    &     &     & \vdots &         &         &     \\
\\
    &     &     &        &         & c_1     & c_2 & c_3 & \ldots & c_{n-1} & c_n \\
c_1 & c_2 & c_3 & \ldots & c_{n-2} & c_{n-1} & c_n \\
\\
    &     &     &        &         &         & c_1 & c_2 & \ldots & c_{n-1} & c_n \\
c_1 & c_2 & c_3 & \ldots & c_{n-2} & c_{n-1} & c_n \\
\end{array}
\]
Since we have a CRCW machine, the instances of $c_i$ can be considered references rather than copies.\\
The rest of the operations are exactly the same as the previous problem, with modifications for CRCW.\\
The AND-problem is constant-time with $n$ processors per ``pairing''.\\
The MAX operation is the same as the MIN operation, which was done in class and is available in the course notes.
It requires $n^2$ processors, comparing each pairing of numbers, returning $1$ if the first is greater, $0$ otherwise.
The AND operation is run on each row. The row containing no zeroes is the MAX.

\newpage
11. Design a parallel algorithm for adding two $n$-bit integers. You algorithm should run in O($log~n$) time
on a CREW PRAM with n processors.\\
NOTE: If your algorithm is EREW, you might want to rethink since I don't know how to do this easily
with out CR.\\
HINT: Use divide and conquer and generalize the induction hypothesis.

\end{document}
